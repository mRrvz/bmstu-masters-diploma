\section*{ВВЕДЕНИЕ}
\addcontentsline{toc}{section}{ВВЕДЕНИЕ}

Необходимость повышения безопасности исполнения приложений, работающих в системах безопасности и обрабатывающих защищаемую информацию, привела к разработке программно-аппаратных решений, создающих доверенные среды исполнения (англ. TEE -- Trusted Execution Environment \cite{tee}) на базе аппаратных средств, доверенных загрузок или аппаратно-программных модулей доверенной загрузки. Intel \cite{intel} и ARM \cite{arm} являются лидерами в этой области. Целью данной работы является изучение существующих реализаций доверенных сред исполнения и разработка метода программной реализации доверенной среды исполнения с помощью виртуализации процессоров архитектуры ARM.

Для достижения поставленной цели необходимо решить следующие задачи:

\begin{itemize}
	\item провести обзор существующих реализаций ДСИ;
	\item описать их достоинства и недостатки;
	\item сформулировать критерии сравнения и сравнить реализации;
	\item изложить особенности метода;
	\item представить формализацию в виде диаграмм IDEF0 и схем алгоритмов;
	\item выполнить проектирование ПО для реализации метода;
	\item обосновать выбор средств программной реализации;
	\item разработать ПО;
	\item провести исследование эффективности и применимости ПО.
\end{itemize}

\pagebreak
