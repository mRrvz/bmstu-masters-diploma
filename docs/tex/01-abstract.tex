\section*{РЕФЕРАТ}
\addcontentsline{toc}{section}{РЕФЕРАТ}

Расчетно-пояснительная записка к выпускной квалификационной работе <<Метод программной реализации доверенной среды исполнения с помощью виртуализации процессоров архитектуры ARM>> содержит 95  страниц, 4 раздела, \totalfigures\ рисунков, \totaltables\ таблиц и список используемых источников из 29 наименований.
% 17+78=95

Ключевые слова: доверенная среда исполнения, виртуализация, безопасность, ARM, TrustZone, операционные системы, системное программирование.

Объект разработки: метод программной реализации доверенной среды исполнения.

Цель работы: разработка метода программной реализации доверенной среды исполнения.
 
В аналитическом разделе представлен обзор реализаций доверенных сред исполнения и описаны их особенности. Сформулированы критерии оценки и проведено сравнение на их основе. Представлен обзор средств виртуализации в процессорных системах ARM. Представлена формализованная постановка задачи на разработку метода программной реализации доверенной среды исполнения с помощью виртуализации процессоров архитектуры ARM.

В конструкторском разделе разработан метод программной реализации доверенной среды исполнения с помощью виртуализации процессоров архитектуры ARM и представлено его формальное описание в виде диаграмм IDEF0 и схем алгоритмов. Выполнено проектирование ПО для реализации данного метода.

В технологическом разделе обоснован выбор средства программной реализации доверенной среды исполнения с помощью виртуализации процессоров архитектуры ARM. Разработано программное обеспечение, реализующее метод доверенной среды исполнения с помощью виртуализации процессоров архитектуры ARM.

В исследовательском разделе проведено исследование эффективности и применимости разработанного программного обеспечения. Выполнено сравнение результатов работы разработанного метода и метода с аппаратной поддержкой доверенной среды исполнения на базе процессоров с архитектурой ARM (ARM TrustZone).

Разработанный метод может найти применение в области серверных и облачных технологий.

\pagebreak
