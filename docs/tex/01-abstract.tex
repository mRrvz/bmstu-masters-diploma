\section*{РЕФЕРАТ}

Расчетно-пояснительная записка \pageref{LastPage} с., \totalfigures\ рис., \totaltables\ табл., 37 ист.

Объектом исследования данной работы является доверенные среды исполнения и их виртуализация. Представлен программный подход к виртуализации доверенной среды исполнения ARM TrustZone, что позволяет безопасно её использовать в виртуальных средах. Целью данной работы является изучение существующих реализаций доверенных сред исполнения и разработка метода программной реализации доверенной среды исполнения с помощью виртуализации процессоров архитектуры ARM.

Для достижения поставленной цели необходимо решить следующие задачи:

\begin{itemize}
	\item провести обзор существующих реализаций ДСИ;
	\item описать их достоинства и недостатки;
	\item сформулировать критерии сравнения и сравнить реализации;
	\item изложить особенности метода;
	\item представить формализацию в виде диаграмм IDEF0 и схем алгоритмов;
	\item выполнить проектирование ПО для реализации метода;
	\item обосновать выбор средств программной реализации;
	\item разработать ПО;
	\item провести исследование эффективности и применимости ПО.
\end{itemize}

Поставленная цель была достигнута: разработан метод программной реализации доверенной среды исполнения с помощью виртуализации процессоров архитектуры ARM -- каждая виртуальная машина может использовать свою собственную, виртуализированную доверенную среду исполнения, которая полностью сохраняет свойства аппаратной реализации.

КЛЮЧЕВЫЕ СЛОВА

\textit{доверенная среда исполнения, виртуализация, ARM, TrustZone, безопасность, операционные системы, системное программирование}

\pagebreak
