\section{Технологический раздел}

В данном разделе обоснован выбор средства программной реализации доверенной среды исполнения с помощью виртуализации процессоров архитектуры ARM. Дано описание ключевого функционала, реализующее метод доверенной среды исполнения с помощью виртуализации процессоров архитектуры ARM.

\subsection{Выбор операционной системы}

В качестве гостевой операционной системы была выбрана ОС Linux с версией ядра 6.17 \cite{linux}. Выбор обоснован тем, что Linux является полностью совместимым с технологий ARM TrustZone, а так же обладает открытым исходным кодом.

В качестве доверенной операционной системы была выбрана ОС OP-TEE версии 4.0. Выбор основан тем, что эта ОС обладает большим объемом документации и открытым исходным кодом.

\subsection{Выбор средств виртуализации}

В качестве гипервизора был выбран KVM (Kernel-Based Virtual Machine) \cite{kvm}, который является частью ядра ОС Linux. Данный выбор обоснован тем, что KVM является технологией с открытым исходным кодом, полностью поддерживается для процессоров архитектуры ARM и является совместимым с ОС Linux.

\subsection{Сборка программного обеспечения}

Разработанное программное обеспечение является модификацией ядра OP-TEE. Для сборки проектов используется специальная утилита make \cite{make}, позволяющая автоматизировать сборку ядра. make является кроссплатформенной системой автоматизации сборки программного обеспечения из исходного кода. make позволяет существенно ускорить процесс сборки проекта. Так, например, при изменении одного исходного файла проекта, заново будет собран в объектный файл лишь этот исходный файл, а не все файлы проекта. В листинге \ref{code:optee-compile} представлен скрипт для сборки ядра OP-TEE.

\begin{code}
	\captionof{listing}{Сборочный скрипт OP-TEE}
	\label{code:optee-compile}
	\inputminted
	[
	frame=single,
	framerule=0.5pt,
	framesep=20pt,
	fontsize=\small,
	tabsize=4,
	linenos,
	numbersep=5pt,
	xleftmargin=10pt,
	]
	{bash}
	{code/optee_build.sh}
\end{code}

В рамках данной работы так же была произведена модификация гипервизора KVM. KVM собирается в составе ядра Linux, поэтому для его сборки необходимо включить опцию \texttt{CONFIG\_KVM} в defconfig ядра. Так же для сборки используется утилита make. В листинге \ref{code:kvm-compile} приведен скрипт для сборки ядра Linux вместе с KVM.

\begin{code}
	\captionof{listing}{Сборочный скрипт ядра Linux вместе с KVM}
	\label{code:kvm-compile}
	\inputminted
	[
	frame=single,
	framerule=0.5pt,
	framesep=20pt,
	fontsize=\small,
	tabsize=4,
	linenos,
	numbersep=5pt,
	xleftmargin=10pt,
	]
	{bash}
	{code/kvm_build.sh}
\end{code}

\subsection{Требования к вычислительной системе}

Для сборки и установки разработанного программного обеспечения требуются следующие библиотеки и утилиты, представленные в таблице \ref{table:deps}.

\begin{table}[]
	\begin{center}
		\caption{Таблица ПО, необходимого для сборки разработанного программного обеспечения}
		\label{table:deps}
		\begin{tabular}{|c|c|}
			\hline
			\bfseries ПО & \bfseries Минимальная версия \\
			\hline
			gcc & 10.2 \\ \hline
			GNU make & 3.80 \\ \hline
			binutils & 2.12 \\ \hline
			util-linux & 2.10o \\ \hline
			module-init-tools & 0.9.10 \\ \hline
			e2fsprogs & 1.41.4 \\ \hline
			jfsutils & 1.1.3 \\ \hline
			reiserfsprogs & 3.6.3 \\ \hline
			xfsprogs & 2.6.0 \\ \hline
			squashfs-tools & 4.0 \\ \hline
			btrfs-progs & 0.18 \\ \hline
			pcmciautils & 004 \\ \hline
			quota-tools & 3.09 \\ \hline
			PPP & 2.4.0 \\ \hline
			isdn4k-utils & 3.1pre1 \\ \hline
			nfs-utils & 1.0.5 \\ \hline
			procps & 3.2.0 \\ \hline
			oprofile & 0.9 \\ \hline
			udev & 081 \\ \hline
			grub & 0.93 \\ \hline
			mcelog & 0.6 \\ \hline
			iptables & 1.4.2 \\ \hline
			openssl, libcrypto & 1.0.0 \\ \hline
			bc & 1.2 \\ \hline
		\end{tabular}
	\end{center}
\end{table}

\subsection{Структура программного обеспечения}

Разработанное ПО представляет из себя модификации исходного кода гипервизора KVM и ОС OP-TEE. Ниже описан ключевой функционал, необходимый для работы системы.

\subsubsection{Функция блокировки потока управления}

При возникновении какого-либо исключения используется функция блокировки потока управления (часть гипервизора KVM). Код необходимых обработчиков исключений был модифицирован таким образом, чтобы управление передавалось необходимым модулям. В листинге  \ref{code:interrupt-handler} приведен модифицированный код обработчика исключений.

\begin{code}
	\captionof{listing}{Модификация кода обработчика исключений}
	\label{code:interrupt-handler}
	\inputminted
	[
	frame=single,
	framerule=0.5pt,
	framesep=20pt,
	fontsize=\small,
	tabsize=4,
	linenos,
	numbersep=5pt,
	xleftmargin=10pt,
	]
	{text}
	{code/interrupt_handler.c}
\end{code}

\subsubsection{Функция переключения контекста}

Переключение контекста между ВМ реализуется в модуле переключения контекста, который исполняется в доверенном мире. Этот модуль предоставляет API, позволяющее переключить контекст из гостевой ВМ в доверенную и наоборот. В листинге \ref{code:change-context} представлен исходный код этой функции.

\begin{code}
	\captionof{listing}{Переключение контекста между двумя виртуальными машинами}
	\label{code:change-context}
	\inputminted
	[
	frame=single,
	framerule=0.5pt,
	framesep=20pt,
	fontsize=\small,
	tabsize=4,
	linenos,
	numbersep=5pt,
	xleftmargin=10pt,
	]
	{text}
	{code/change_context.c}
\end{code}

\subsubsection{Функция защищенного отображения памяти}

Модуль отображения памяти, который так же исполняется в контексте доверенной ОС, отвечает за функции защищенного отображения памяти между виртуальными машинами. В листинге \ref{code:memory} представлена функция, позволяющая обходить страницы памяти ВМ и в зависимости от входных параметров выполнять действия над ними.

\begin{code}
	\captionof{listing}{Отображение памяти, реализованное на уровне доверенной ОС}
	\label{code:memory}
	\inputminted
	[
	frame=single,
	framerule=0.5pt,
	framesep=20pt,
	fontsize=\small,
	tabsize=4,
	linenos,
	numbersep=5pt,
	xleftmargin=10pt,
	]
	{text}
	{code/pgtable.c}
\end{code}

\subsection*{Вывод}

В данном разделе были описаны средства разработки программного обеспечения, требования к ПО и системе для сборки этого ПО. Было приведено описание ключевых функций разработанного метода:

\begin{itemize}
	\item функция блокировки потока управления, реализованная на уровне KVM;
	\item функция переключения контекста между ВМ;
	\item функция защищенного отображения памяти ВМ.
\end{itemize}

\pagebreak
