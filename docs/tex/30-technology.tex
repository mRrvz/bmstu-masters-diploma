\section{Технологическая часть}

В данном разделе обоснован выбор средства программной реализации доверенной среды исполнения с помощью виртуализации процессоров архитектуры ARM.
Разработано программное обеспечение, реализующее метод доверенной среды исполнения с помощью виртуализации процессоров архитектуры ARM.

\subsection{Выбор операционной системы}

В качестве операционной системы была выбрана ОС Linux с версией ядра 6.17 \cite{linux}. Выбор обоснован тем, что Linux является полностью совместимым с технологий ARM TrustZone, а так же обладает открытым исходным кодом.

\subsection{Выбор средств виртуализации}

В качестве гипервизора был выбран KVM (Kernel-Based Virtual Machine) \cite{kvm}, который является частью ядра ОС Linux. Данный выбор обоснован тем, что KVM является технологией с открытым исходным кодом, полностью поддерживается для процессоров архитектуры ARM и является совместимым с ОС Linux.

В качестве эмулятора аппаратного обеспечения была выбран QEMU (Quick Emulator) версии 8.2.1 \cite{qemu}. Выбор QEMU обоснован тем, что данное ПО обладает открытым исходным кодом, а так же полностью совместимым и заточенным под работу в связке с KVM \cite{kvm}.

\subsection{Сборка программного обеспечения}

Разработанное программное обеспечение является модификацией эмулятора QEMU. Для сборки проекта используется специальная утилита make \cite{make}, позволяющая автоматизировать сборку ядра. make является кроссплатформенной системой автоматизации сборки программного обеспечения из исходного кода. make позволяет существенно ускорить процесс сборки проекта. Так, например, при изменении одного исходного файла проекта, заново будет собран в объектный файл лишь этот исходный файл, а не все файлы проекта. В листинге \ref{code:qemu-compile} представлен скрипт для сборки QEMU.

\begin{code}
	\captionof{listing}{Сборочный скрипт QEMU}
	\label{code:qemu-compile}
	\inputminted
	[
	frame=single,
	framerule=0.5pt,
	framesep=20pt,
	fontsize=\small,
	tabsize=4,
	linenos,
	numbersep=5pt,
	xleftmargin=10pt,
	]
	{bash}
	{code/qemu_build.sh}
\end{code}

В рамках данной работы было разработано доверенное приложение ARM TrustZone, отвечающее за проверку целостности загружаемого образа виртуальной машины. Для его сборки так же используется утилита make. В листинге \ref{code:trustapplet-compile} приведен скрипт для сборки доверенного приложения.

\begin{code}
	\captionof{listing}{Сборочный скрипт доверенного приложения}
	\label{code:trustapplet-compile}
	\inputminted
	[
	frame=single,
	framerule=0.5pt,
	framesep=20pt,
	fontsize=\small,
	tabsize=4,
	linenos,
	numbersep=5pt,
	xleftmargin=10pt,
	]
	{bash}
	{code/trustapplet_build.sh}
\end{code}

\subsection{Требования к вычислительной системе}

Для сборки и установки разработанного программного обеспечения требуются следующие библиотеки и утилиты, представленные в таблице \ref{table:dependencies}.

\begin{table}[!htb]
	\label{table:dependencies}
	\begin{center}
		\caption{Таблица зависимостей, необходимого для сборки разработанного ПО}
		\begin{tabular}{|c|c|}
			\hline
			\bfseries ПО & \bfseries Минимальная версия \\
			\hline
			gcc & 12.4.0 \\ \hline
			GNU make & 4.3 \\ \hline
			Autoconf & 2.71 \\ \hline
			Automake & 1.16.4 \\ \hline
			Libtool & 2.4.6 \\ \hline
			Bison & 3.8.2 \\ \hline
			Flex & 2.6.4 \\ \hline
			libcap & 2.42 \\ \hline
			ncurses & 6.3 \\
			\hline
		\end{tabular}
	\end{center}
\end{table}

\subsection{Структура программного обеспечения}

Разработанное ПО представляет из себя !!!. Ниже описывается эта функция  !!!

\subsubsection{Модификация гипервизора}

\subsubsection{Обработчик переключения контекста}

\subsubsection{Функция обработки элементов таблицы страниц}

\subsection*{Вывод}

В данном разделе были описаны средства разработки программного обеспечения и требования к ПО. Была приведена структура разработанного ПО.

\pagebreak
